% Options for packages loaded elsewhere
\PassOptionsToPackage{unicode}{hyperref}
\PassOptionsToPackage{hyphens}{url}
%
\documentclass[
]{article}
\usepackage{amsmath,amssymb}
\usepackage{iftex}
\ifPDFTeX
  \usepackage[T1]{fontenc}
  \usepackage[utf8]{inputenc}
  \usepackage{textcomp} % provide euro and other symbols
\else % if luatex or xetex
  \usepackage{unicode-math} % this also loads fontspec
  \defaultfontfeatures{Scale=MatchLowercase}
  \defaultfontfeatures[\rmfamily]{Ligatures=TeX,Scale=1}
\fi
\usepackage{lmodern}
\ifPDFTeX\else
  % xetex/luatex font selection
\fi
% Use upquote if available, for straight quotes in verbatim environments
\IfFileExists{upquote.sty}{\usepackage{upquote}}{}
\IfFileExists{microtype.sty}{% use microtype if available
  \usepackage[]{microtype}
  \UseMicrotypeSet[protrusion]{basicmath} % disable protrusion for tt fonts
}{}
\makeatletter
\@ifundefined{KOMAClassName}{% if non-KOMA class
  \IfFileExists{parskip.sty}{%
    \usepackage{parskip}
  }{% else
    \setlength{\parindent}{0pt}
    \setlength{\parskip}{6pt plus 2pt minus 1pt}}
}{% if KOMA class
  \KOMAoptions{parskip=half}}
\makeatother
\usepackage{xcolor}
\usepackage[margin=1in]{geometry}
\usepackage{color}
\usepackage{fancyvrb}
\newcommand{\VerbBar}{|}
\newcommand{\VERB}{\Verb[commandchars=\\\{\}]}
\DefineVerbatimEnvironment{Highlighting}{Verbatim}{commandchars=\\\{\}}
% Add ',fontsize=\small' for more characters per line
\usepackage{framed}
\definecolor{shadecolor}{RGB}{248,248,248}
\newenvironment{Shaded}{\begin{snugshade}}{\end{snugshade}}
\newcommand{\AlertTok}[1]{\textcolor[rgb]{0.94,0.16,0.16}{#1}}
\newcommand{\AnnotationTok}[1]{\textcolor[rgb]{0.56,0.35,0.01}{\textbf{\textit{#1}}}}
\newcommand{\AttributeTok}[1]{\textcolor[rgb]{0.13,0.29,0.53}{#1}}
\newcommand{\BaseNTok}[1]{\textcolor[rgb]{0.00,0.00,0.81}{#1}}
\newcommand{\BuiltInTok}[1]{#1}
\newcommand{\CharTok}[1]{\textcolor[rgb]{0.31,0.60,0.02}{#1}}
\newcommand{\CommentTok}[1]{\textcolor[rgb]{0.56,0.35,0.01}{\textit{#1}}}
\newcommand{\CommentVarTok}[1]{\textcolor[rgb]{0.56,0.35,0.01}{\textbf{\textit{#1}}}}
\newcommand{\ConstantTok}[1]{\textcolor[rgb]{0.56,0.35,0.01}{#1}}
\newcommand{\ControlFlowTok}[1]{\textcolor[rgb]{0.13,0.29,0.53}{\textbf{#1}}}
\newcommand{\DataTypeTok}[1]{\textcolor[rgb]{0.13,0.29,0.53}{#1}}
\newcommand{\DecValTok}[1]{\textcolor[rgb]{0.00,0.00,0.81}{#1}}
\newcommand{\DocumentationTok}[1]{\textcolor[rgb]{0.56,0.35,0.01}{\textbf{\textit{#1}}}}
\newcommand{\ErrorTok}[1]{\textcolor[rgb]{0.64,0.00,0.00}{\textbf{#1}}}
\newcommand{\ExtensionTok}[1]{#1}
\newcommand{\FloatTok}[1]{\textcolor[rgb]{0.00,0.00,0.81}{#1}}
\newcommand{\FunctionTok}[1]{\textcolor[rgb]{0.13,0.29,0.53}{\textbf{#1}}}
\newcommand{\ImportTok}[1]{#1}
\newcommand{\InformationTok}[1]{\textcolor[rgb]{0.56,0.35,0.01}{\textbf{\textit{#1}}}}
\newcommand{\KeywordTok}[1]{\textcolor[rgb]{0.13,0.29,0.53}{\textbf{#1}}}
\newcommand{\NormalTok}[1]{#1}
\newcommand{\OperatorTok}[1]{\textcolor[rgb]{0.81,0.36,0.00}{\textbf{#1}}}
\newcommand{\OtherTok}[1]{\textcolor[rgb]{0.56,0.35,0.01}{#1}}
\newcommand{\PreprocessorTok}[1]{\textcolor[rgb]{0.56,0.35,0.01}{\textit{#1}}}
\newcommand{\RegionMarkerTok}[1]{#1}
\newcommand{\SpecialCharTok}[1]{\textcolor[rgb]{0.81,0.36,0.00}{\textbf{#1}}}
\newcommand{\SpecialStringTok}[1]{\textcolor[rgb]{0.31,0.60,0.02}{#1}}
\newcommand{\StringTok}[1]{\textcolor[rgb]{0.31,0.60,0.02}{#1}}
\newcommand{\VariableTok}[1]{\textcolor[rgb]{0.00,0.00,0.00}{#1}}
\newcommand{\VerbatimStringTok}[1]{\textcolor[rgb]{0.31,0.60,0.02}{#1}}
\newcommand{\WarningTok}[1]{\textcolor[rgb]{0.56,0.35,0.01}{\textbf{\textit{#1}}}}
\usepackage{graphicx}
\makeatletter
\newsavebox\pandoc@box
\newcommand*\pandocbounded[1]{% scales image to fit in text height/width
  \sbox\pandoc@box{#1}%
  \Gscale@div\@tempa{\textheight}{\dimexpr\ht\pandoc@box+\dp\pandoc@box\relax}%
  \Gscale@div\@tempb{\linewidth}{\wd\pandoc@box}%
  \ifdim\@tempb\p@<\@tempa\p@\let\@tempa\@tempb\fi% select the smaller of both
  \ifdim\@tempa\p@<\p@\scalebox{\@tempa}{\usebox\pandoc@box}%
  \else\usebox{\pandoc@box}%
  \fi%
}
% Set default figure placement to htbp
\def\fps@figure{htbp}
\makeatother
\setlength{\emergencystretch}{3em} % prevent overfull lines
\providecommand{\tightlist}{%
  \setlength{\itemsep}{0pt}\setlength{\parskip}{0pt}}
\setcounter{secnumdepth}{-\maxdimen} % remove section numbering
\usepackage{bookmark}
\IfFileExists{xurl.sty}{\usepackage{xurl}}{} % add URL line breaks if available
\urlstyle{same}
\hypersetup{
  pdftitle={Assignment\_1.rmd},
  pdfauthor={Andrew Shenal},
  hidelinks,
  pdfcreator={LaTeX via pandoc}}

\title{Assignment\_1.rmd}
\author{Andrew Shenal}
\date{2025-09-04}

\begin{document}
\maketitle

This cell loads the tidyverse package which is very helpful for data
manipulation in R

\begin{Shaded}
\begin{Highlighting}[]
\CommentTok{\# Set up key functions for R and manipulating data}
\FunctionTok{library}\NormalTok{(tidyverse)}
\end{Highlighting}
\end{Shaded}

The code below imports the data from the R project then uses two methods
for basic descriptive statistics. Sapply (found in the R documentation)
is used to acquire mean, median, and range variables. Summary() is used
across the whole dataset as an alternative method of acquiring
descriptive statistics.

\begin{Shaded}
\begin{Highlighting}[]
\CommentTok{\# Data acquired from Kaggle website: Berk Erisen {-} "Wind Turbine Scada Dataset: 2018 Scada Data of a Wind Turbine in Turkey".}
\NormalTok{Wind }\OtherTok{\textless{}{-}} \FunctionTok{read\_csv}\NormalTok{(}\StringTok{"./archive/T1.csv"}\NormalTok{)}
\end{Highlighting}
\end{Shaded}

\begin{verbatim}
## Rows: 50530 Columns: 5
## -- Column specification --------------------------------------------------------
## Delimiter: ","
## chr (1): Date/Time
## dbl (4): LV ActivePower (kW), Wind Speed (m/s), Theoretical_Power_Curve (KWh...
## 
## i Use `spec()` to retrieve the full column specification for this data.
## i Specify the column types or set `show_col_types = FALSE` to quiet this message.
\end{verbatim}

\begin{Shaded}
\begin{Highlighting}[]
\CommentTok{\# Descriptive Statistics using sapply}
\NormalTok{means }\OtherTok{\textless{}{-}} \FunctionTok{sapply}\NormalTok{(Wind, mean, }\AttributeTok{na.rm =} \ConstantTok{TRUE}\NormalTok{)}
\NormalTok{median }\OtherTok{\textless{}{-}} \FunctionTok{sapply}\NormalTok{(Wind, median, }\AttributeTok{na.rm =} \ConstantTok{TRUE}\NormalTok{)}
\NormalTok{range }\OtherTok{\textless{}{-}} \FunctionTok{sapply}\NormalTok{(Wind, range, }\AttributeTok{na.rm =} \ConstantTok{TRUE}\NormalTok{)}

\FunctionTok{print}\NormalTok{(means)}
\end{Highlighting}
\end{Shaded}

\begin{verbatim}
##                     Date/Time           LV ActivePower (kW) 
##                            NA                   1307.684332 
##              Wind Speed (m/s) Theoretical_Power_Curve (KWh) 
##                      7.557952                   1492.175463 
##            Wind Direction (°) 
##                    123.687559
\end{verbatim}

\begin{Shaded}
\begin{Highlighting}[]
\FunctionTok{print}\NormalTok{(median)}
\end{Highlighting}
\end{Shaded}

\begin{verbatim}
##                     Date/Time           LV ActivePower (kW) 
##                            NA                    825.838074 
##              Wind Speed (m/s) Theoretical_Power_Curve (KWh) 
##                      7.104594                   1063.776283 
##            Wind Direction (°) 
##                     73.712978
\end{verbatim}

\begin{Shaded}
\begin{Highlighting}[]
\FunctionTok{print}\NormalTok{(range)}
\end{Highlighting}
\end{Shaded}

\begin{verbatim}
##      Date/Time          LV ActivePower (kW) Wind Speed (m/s)  
## [1,] "01 01 2018 00:00" "-2.47140502929687" "0"               
## [2,] "31 12 2018 23:50" "3618.73291015625"  "25.2060108184814"
##      Theoretical_Power_Curve (KWh) Wind Direction (°)
## [1,] "0"                           "0"               
## [2,] "3600"                        "359.997589111328"
\end{verbatim}

\begin{Shaded}
\begin{Highlighting}[]
\CommentTok{\# Descriptive Statistics using summary}
\FunctionTok{summary}\NormalTok{(Wind)}
\end{Highlighting}
\end{Shaded}

\begin{verbatim}
##   Date/Time         LV ActivePower (kW) Wind Speed (m/s)
##  Length:50530       Min.   :  -2.471    Min.   : 0.000  
##  Class :character   1st Qu.:  50.678    1st Qu.: 4.201  
##  Mode  :character   Median : 825.838    Median : 7.105  
##                     Mean   :1307.684    Mean   : 7.558  
##                     3rd Qu.:2482.508    3rd Qu.:10.300  
##                     Max.   :3618.733    Max.   :25.206  
##  Theoretical_Power_Curve (KWh) Wind Direction (°)
##  Min.   :   0.0                Min.   :  0.00    
##  1st Qu.: 161.3                1st Qu.: 49.32    
##  Median :1063.8                Median : 73.71    
##  Mean   :1492.2                Mean   :123.69    
##  3rd Qu.:2965.0                3rd Qu.:201.70    
##  Max.   :3600.0                Max.   :360.00
\end{verbatim}

I converted the wind speed variable which is in m/s to feet per second
by multiplying the column by 3.28084.

\begin{Shaded}
\begin{Highlighting}[]
\CommentTok{\# Variable Transformation}
\NormalTok{Wind }\SpecialCharTok{\%\textgreater{}\%} 
  \FunctionTok{mutate}\NormalTok{(}\StringTok{\textasciigrave{}}\AttributeTok{Wind Speed (feet/s)}\StringTok{\textasciigrave{}} \OtherTok{=} \StringTok{\textasciigrave{}}\AttributeTok{Wind Speed (m/s)}\StringTok{\textasciigrave{}}\SpecialCharTok{*}\FloatTok{3.28084}\NormalTok{) }\SpecialCharTok{\%\textgreater{}\%} 
  \FunctionTok{head}\NormalTok{()}
\end{Highlighting}
\end{Shaded}

\begin{verbatim}
## # A tibble: 6 x 6
##   `Date/Time`    `LV ActivePower (kW)` `Wind Speed (m/s)` Theoretical_Power_Cu~1
##   <chr>                          <dbl>              <dbl>                  <dbl>
## 1 01 01 2018 00~                  380.               5.31                   416.
## 2 01 01 2018 00~                  454.               5.67                   520.
## 3 01 01 2018 00~                  306.               5.22                   391.
## 4 01 01 2018 00~                  420.               5.66                   516.
## 5 01 01 2018 00~                  381.               5.58                   492.
## 6 01 01 2018 00~                  402.               5.60                   499.
## # i abbreviated name: 1: `Theoretical_Power_Curve (KWh)`
## # i 2 more variables: `Wind Direction (°)` <dbl>, `Wind Speed (feet/s)` <dbl>
\end{verbatim}

Below is a scatterplot containing wind speed on the x axis and power
generated from the wind turbine on the y axis.

\begin{Shaded}
\begin{Highlighting}[]
\FunctionTok{plot}\NormalTok{(Wind}\SpecialCharTok{$}\StringTok{\textasciigrave{}}\AttributeTok{Wind Speed (m/s)}\StringTok{\textasciigrave{}}\NormalTok{, Wind}\SpecialCharTok{$}\StringTok{\textasciigrave{}}\AttributeTok{LV ActivePower (kW)}\StringTok{\textasciigrave{}}\NormalTok{,}
     \AttributeTok{xlab =} \StringTok{"Wind Speed (m/s)"}\NormalTok{, }\AttributeTok{ylab =} \StringTok{"LV ActivePower (kW)"}\NormalTok{)}
\end{Highlighting}
\end{Shaded}

\pandocbounded{\includegraphics[keepaspectratio]{Assignment_1_files/figure-latex/unnamed-chunk-4-1.pdf}}

Below is a histogram plot of the power created over the course of the
data collection period.

\begin{Shaded}
\begin{Highlighting}[]
\FunctionTok{hist}\NormalTok{(Wind}\SpecialCharTok{$}\StringTok{\textasciigrave{}}\AttributeTok{LV ActivePower (kW)}\StringTok{\textasciigrave{}}\NormalTok{,}
     \AttributeTok{xlab =} \StringTok{"LV ActivePower (kW)"}\NormalTok{)}
\end{Highlighting}
\end{Shaded}

\pandocbounded{\includegraphics[keepaspectratio]{Assignment_1_files/figure-latex/unnamed-chunk-5-1.pdf}}

\end{document}
