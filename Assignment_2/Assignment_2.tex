% Options for packages loaded elsewhere
\PassOptionsToPackage{unicode}{hyperref}
\PassOptionsToPackage{hyphens}{url}
%
\documentclass[
]{article}
\usepackage{amsmath,amssymb}
\usepackage{iftex}
\ifPDFTeX
  \usepackage[T1]{fontenc}
  \usepackage[utf8]{inputenc}
  \usepackage{textcomp} % provide euro and other symbols
\else % if luatex or xetex
  \usepackage{unicode-math} % this also loads fontspec
  \defaultfontfeatures{Scale=MatchLowercase}
  \defaultfontfeatures[\rmfamily]{Ligatures=TeX,Scale=1}
\fi
\usepackage{lmodern}
\ifPDFTeX\else
  % xetex/luatex font selection
\fi
% Use upquote if available, for straight quotes in verbatim environments
\IfFileExists{upquote.sty}{\usepackage{upquote}}{}
\IfFileExists{microtype.sty}{% use microtype if available
  \usepackage[]{microtype}
  \UseMicrotypeSet[protrusion]{basicmath} % disable protrusion for tt fonts
}{}
\makeatletter
\@ifundefined{KOMAClassName}{% if non-KOMA class
  \IfFileExists{parskip.sty}{%
    \usepackage{parskip}
  }{% else
    \setlength{\parindent}{0pt}
    \setlength{\parskip}{6pt plus 2pt minus 1pt}}
}{% if KOMA class
  \KOMAoptions{parskip=half}}
\makeatother
\usepackage{xcolor}
\usepackage[margin=1in]{geometry}
\usepackage{color}
\usepackage{fancyvrb}
\newcommand{\VerbBar}{|}
\newcommand{\VERB}{\Verb[commandchars=\\\{\}]}
\DefineVerbatimEnvironment{Highlighting}{Verbatim}{commandchars=\\\{\}}
% Add ',fontsize=\small' for more characters per line
\usepackage{framed}
\definecolor{shadecolor}{RGB}{248,248,248}
\newenvironment{Shaded}{\begin{snugshade}}{\end{snugshade}}
\newcommand{\AlertTok}[1]{\textcolor[rgb]{0.94,0.16,0.16}{#1}}
\newcommand{\AnnotationTok}[1]{\textcolor[rgb]{0.56,0.35,0.01}{\textbf{\textit{#1}}}}
\newcommand{\AttributeTok}[1]{\textcolor[rgb]{0.13,0.29,0.53}{#1}}
\newcommand{\BaseNTok}[1]{\textcolor[rgb]{0.00,0.00,0.81}{#1}}
\newcommand{\BuiltInTok}[1]{#1}
\newcommand{\CharTok}[1]{\textcolor[rgb]{0.31,0.60,0.02}{#1}}
\newcommand{\CommentTok}[1]{\textcolor[rgb]{0.56,0.35,0.01}{\textit{#1}}}
\newcommand{\CommentVarTok}[1]{\textcolor[rgb]{0.56,0.35,0.01}{\textbf{\textit{#1}}}}
\newcommand{\ConstantTok}[1]{\textcolor[rgb]{0.56,0.35,0.01}{#1}}
\newcommand{\ControlFlowTok}[1]{\textcolor[rgb]{0.13,0.29,0.53}{\textbf{#1}}}
\newcommand{\DataTypeTok}[1]{\textcolor[rgb]{0.13,0.29,0.53}{#1}}
\newcommand{\DecValTok}[1]{\textcolor[rgb]{0.00,0.00,0.81}{#1}}
\newcommand{\DocumentationTok}[1]{\textcolor[rgb]{0.56,0.35,0.01}{\textbf{\textit{#1}}}}
\newcommand{\ErrorTok}[1]{\textcolor[rgb]{0.64,0.00,0.00}{\textbf{#1}}}
\newcommand{\ExtensionTok}[1]{#1}
\newcommand{\FloatTok}[1]{\textcolor[rgb]{0.00,0.00,0.81}{#1}}
\newcommand{\FunctionTok}[1]{\textcolor[rgb]{0.13,0.29,0.53}{\textbf{#1}}}
\newcommand{\ImportTok}[1]{#1}
\newcommand{\InformationTok}[1]{\textcolor[rgb]{0.56,0.35,0.01}{\textbf{\textit{#1}}}}
\newcommand{\KeywordTok}[1]{\textcolor[rgb]{0.13,0.29,0.53}{\textbf{#1}}}
\newcommand{\NormalTok}[1]{#1}
\newcommand{\OperatorTok}[1]{\textcolor[rgb]{0.81,0.36,0.00}{\textbf{#1}}}
\newcommand{\OtherTok}[1]{\textcolor[rgb]{0.56,0.35,0.01}{#1}}
\newcommand{\PreprocessorTok}[1]{\textcolor[rgb]{0.56,0.35,0.01}{\textit{#1}}}
\newcommand{\RegionMarkerTok}[1]{#1}
\newcommand{\SpecialCharTok}[1]{\textcolor[rgb]{0.81,0.36,0.00}{\textbf{#1}}}
\newcommand{\SpecialStringTok}[1]{\textcolor[rgb]{0.31,0.60,0.02}{#1}}
\newcommand{\StringTok}[1]{\textcolor[rgb]{0.31,0.60,0.02}{#1}}
\newcommand{\VariableTok}[1]{\textcolor[rgb]{0.00,0.00,0.00}{#1}}
\newcommand{\VerbatimStringTok}[1]{\textcolor[rgb]{0.31,0.60,0.02}{#1}}
\newcommand{\WarningTok}[1]{\textcolor[rgb]{0.56,0.35,0.01}{\textbf{\textit{#1}}}}
\usepackage{graphicx}
\makeatletter
\def\maxwidth{\ifdim\Gin@nat@width>\linewidth\linewidth\else\Gin@nat@width\fi}
\def\maxheight{\ifdim\Gin@nat@height>\textheight\textheight\else\Gin@nat@height\fi}
\makeatother
% Scale images if necessary, so that they will not overflow the page
% margins by default, and it is still possible to overwrite the defaults
% using explicit options in \includegraphics[width, height, ...]{}
\setkeys{Gin}{width=\maxwidth,height=\maxheight,keepaspectratio}
% Set default figure placement to htbp
\makeatletter
\def\fps@figure{htbp}
\makeatother
\setlength{\emergencystretch}{3em} % prevent overfull lines
\providecommand{\tightlist}{%
  \setlength{\itemsep}{0pt}\setlength{\parskip}{0pt}}
\setcounter{secnumdepth}{-\maxdimen} % remove section numbering
\ifLuaTeX
  \usepackage{selnolig}  % disable illegal ligatures
\fi
\usepackage{bookmark}
\IfFileExists{xurl.sty}{\usepackage{xurl}}{} % add URL line breaks if available
\urlstyle{same}
\hypersetup{
  pdftitle={Assignment\_2},
  pdfauthor={Andrew Shenal},
  hidelinks,
  pdfcreator={LaTeX via pandoc}}

\title{Assignment\_2}
\author{Andrew Shenal}
\date{2025-09-25}

\begin{document}
\maketitle

\begin{Shaded}
\begin{Highlighting}[]
\FunctionTok{library}\NormalTok{(caret)}
\end{Highlighting}
\end{Shaded}

\begin{verbatim}
## Warning: package 'caret' was built under R version 4.4.3
\end{verbatim}

\begin{verbatim}
## Loading required package: ggplot2
\end{verbatim}

\begin{verbatim}
## Loading required package: lattice
\end{verbatim}

\begin{Shaded}
\begin{Highlighting}[]
\FunctionTok{library}\NormalTok{(FNN)}
\end{Highlighting}
\end{Shaded}

\begin{verbatim}
## Warning: package 'FNN' was built under R version 4.4.3
\end{verbatim}

\begin{Shaded}
\begin{Highlighting}[]
\FunctionTok{library}\NormalTok{(tidyverse)}
\end{Highlighting}
\end{Shaded}

\begin{verbatim}
## -- Attaching core tidyverse packages ------------------------ tidyverse 2.0.0 --
## v dplyr     1.1.4     v readr     2.1.5
## v forcats   1.0.0     v stringr   1.5.1
## v lubridate 1.9.4     v tibble    3.2.1
## v purrr     1.0.4     v tidyr     1.3.1
\end{verbatim}

\begin{verbatim}
## -- Conflicts ------------------------------------------ tidyverse_conflicts() --
## x dplyr::filter() masks stats::filter()
## x dplyr::lag()    masks stats::lag()
## x purrr::lift()   masks caret::lift()
## i Use the conflicted package (<http://conflicted.r-lib.org/>) to force all conflicts to become errors
\end{verbatim}

\begin{Shaded}
\begin{Highlighting}[]
\NormalTok{Bank }\OtherTok{\textless{}{-}} \FunctionTok{read.csv}\NormalTok{(}\StringTok{"./UniversalBank.csv"}\NormalTok{)}

\CommentTok{\# Normalize dataset and remove education column}
\NormalTok{norm\_model }\OtherTok{\textless{}{-}} \FunctionTok{preProcess}\NormalTok{(Bank, }\AttributeTok{method =} \FunctionTok{c}\NormalTok{(}\StringTok{\textquotesingle{}range\textquotesingle{}}\NormalTok{))}
\NormalTok{Bank\_normalized }\OtherTok{\textless{}{-}} \FunctionTok{predict}\NormalTok{(norm\_model,Bank) }\SpecialCharTok{\%\textgreater{}\%} 
  \FunctionTok{select}\NormalTok{(}\SpecialCharTok{{-}}\NormalTok{Education)}

\CommentTok{\# Create dummy variables for education}
\NormalTok{Bank}\SpecialCharTok{$}\NormalTok{Education }\OtherTok{\textless{}{-}} \FunctionTok{as.factor}\NormalTok{(Bank}\SpecialCharTok{$}\NormalTok{Education)}

\NormalTok{dummy\_model }\OtherTok{\textless{}{-}} \FunctionTok{dummyVars}\NormalTok{(}\SpecialCharTok{\textasciitilde{}}\NormalTok{ Education, }\AttributeTok{data =}\NormalTok{ Bank)}
\NormalTok{dummy\_vars }\OtherTok{\textless{}{-}} \FunctionTok{as.data.frame}\NormalTok{(}\FunctionTok{predict}\NormalTok{(dummy\_model, Bank))}

\CommentTok{\# Add dummy vars to the dataset}
\NormalTok{Bank\_normalized }\OtherTok{\textless{}{-}} \FunctionTok{cbind}\NormalTok{(dummy\_vars, Bank\_normalized)}
\end{Highlighting}
\end{Shaded}

\begin{Shaded}
\begin{Highlighting}[]
\CommentTok{\# Partition the data into 60\% training 40\% testing}
\NormalTok{Train\_index }\OtherTok{\textless{}{-}} \FunctionTok{createDataPartition}\NormalTok{(Bank\_normalized}\SpecialCharTok{$}\NormalTok{Personal.Loan, }\AttributeTok{p=}\FloatTok{0.60}\NormalTok{, }\AttributeTok{list=}\ConstantTok{FALSE}\NormalTok{)}

\NormalTok{Train }\OtherTok{\textless{}{-}}\NormalTok{ Bank\_normalized[Train\_index,]}
\NormalTok{Test }\OtherTok{\textless{}{-}}\NormalTok{ Bank\_normalized[}\SpecialCharTok{{-}}\NormalTok{Train\_index,]}

\CommentTok{\# Select the correct columns for training and testing variables}
\NormalTok{Train\_Predictors}\OtherTok{\textless{}{-}}\NormalTok{Train[,}\SpecialCharTok{{-}}\FunctionTok{c}\NormalTok{(}\DecValTok{4}\NormalTok{, }\DecValTok{8}\NormalTok{, }\DecValTok{12}\NormalTok{)]}
\NormalTok{Test\_Predictors}\OtherTok{\textless{}{-}}\NormalTok{Test[,}\SpecialCharTok{{-}}\FunctionTok{c}\NormalTok{(}\DecValTok{4}\NormalTok{, }\DecValTok{8}\NormalTok{, }\DecValTok{12}\NormalTok{)] }

\NormalTok{Train\_labels }\OtherTok{\textless{}{-}}\NormalTok{Train[,}\DecValTok{12}\NormalTok{] }
\NormalTok{Test\_labels  }\OtherTok{\textless{}{-}}\NormalTok{Test[,}\DecValTok{12}\NormalTok{] }
\end{Highlighting}
\end{Shaded}

\begin{Shaded}
\begin{Highlighting}[]
\CommentTok{\# Create a data frame for example customer in question 1}
\NormalTok{Customer }\OtherTok{\textless{}{-}} \FunctionTok{data.frame}\NormalTok{(}
  \AttributeTok{ID =} \DecValTok{0}\NormalTok{,}
  \AttributeTok{Age =} \DecValTok{40}\NormalTok{,}
  \AttributeTok{Experience =} \DecValTok{10}\NormalTok{,}
  \AttributeTok{Income =} \DecValTok{84}\NormalTok{,}
  \AttributeTok{ZIP.Code =} \DecValTok{0}\NormalTok{,}
  \AttributeTok{Family =} \DecValTok{2}\NormalTok{,}
  \AttributeTok{CCAvg =} \DecValTok{2}\NormalTok{,}
  \AttributeTok{Education =} \DecValTok{0}\NormalTok{,}
  \AttributeTok{Mortgage =} \DecValTok{0}\NormalTok{,}
  \AttributeTok{Personal.Loan =} \DecValTok{0}\NormalTok{,}
  \AttributeTok{Securities.Account =} \DecValTok{0}\NormalTok{,}
  \AttributeTok{CD.Account =} \DecValTok{0}\NormalTok{,}
  \AttributeTok{Online =} \DecValTok{1}\NormalTok{,}
  \AttributeTok{CreditCard =} \DecValTok{1}
\NormalTok{)}

\CommentTok{\# Normalize the customer using the previous normalization model (and add in the correct education values)}
\NormalTok{Customer\_normalized }\OtherTok{\textless{}{-}} \FunctionTok{predict}\NormalTok{(norm\_model, Customer)}

\NormalTok{Customer\_normalized }\OtherTok{\textless{}{-}}\NormalTok{ Customer\_normalized }\SpecialCharTok{\%\textgreater{}\%} 
  \FunctionTok{mutate}\NormalTok{(}\AttributeTok{Eudcation.1 =} \DecValTok{0}\NormalTok{,}
         \AttributeTok{Education.2 =} \DecValTok{1}\NormalTok{,}
         \AttributeTok{Education.3 =} \DecValTok{0}\NormalTok{) }\SpecialCharTok{\%\textgreater{}\%} 
  \FunctionTok{select}\NormalTok{(}\SpecialCharTok{{-}}\NormalTok{Education)}

\CommentTok{\# Select the correct variables for training}
\NormalTok{Customer\_Predictors }\OtherTok{\textless{}{-}}\NormalTok{ Customer\_normalized[}\SpecialCharTok{{-}}\FunctionTok{c}\NormalTok{(}\DecValTok{1}\NormalTok{,}\DecValTok{5}\NormalTok{,}\DecValTok{9}\NormalTok{)]}
\end{Highlighting}
\end{Shaded}

\begin{Shaded}
\begin{Highlighting}[]
\CommentTok{\# Create the KNN model using the customer as the predictor}
\NormalTok{Predicted\_Test\_labels }\OtherTok{\textless{}{-}}\FunctionTok{knn}\NormalTok{(Train\_Predictors, }
\NormalTok{                           Customer\_Predictors, }
                           \AttributeTok{cl=}\NormalTok{Train\_labels, }
                           \AttributeTok{k=}\DecValTok{1}\NormalTok{)}

\FunctionTok{head}\NormalTok{(Predicted\_Test\_labels)}
\end{Highlighting}
\end{Shaded}

\begin{verbatim}
## [1] 0
## Levels: 0
\end{verbatim}

\begin{enumerate}
\def\labelenumi{\arabic{enumi}.}
\tightlist
\item
  The output is zero meaning that the customer would be classified as
  not accepting the loan.
\end{enumerate}

\begin{Shaded}
\begin{Highlighting}[]
\CommentTok{\# creeate a data frame that will hold the accuracy values for all of the K values tested}
\NormalTok{accuracy.df }\OtherTok{\textless{}{-}} \FunctionTok{data.frame}\NormalTok{(}\AttributeTok{k =} \FunctionTok{seq}\NormalTok{(}\DecValTok{1}\NormalTok{, }\DecValTok{20}\NormalTok{, }\DecValTok{1}\NormalTok{), }\AttributeTok{accuracy =} \FunctionTok{rep}\NormalTok{(}\DecValTok{0}\NormalTok{, }\DecValTok{20}\NormalTok{))}

\CommentTok{\# compute knn for different k on validation.}
\ControlFlowTok{for}\NormalTok{(i }\ControlFlowTok{in} \DecValTok{1}\SpecialCharTok{:}\DecValTok{20}\NormalTok{)\{}
\NormalTok{  knn.pred }\OtherTok{\textless{}{-}} \FunctionTok{knn}\NormalTok{(Train\_Predictors, Test\_Predictors, }
                  \AttributeTok{cl =}\NormalTok{ Train\_labels, }\AttributeTok{k =}\NormalTok{ i)}
\NormalTok{  accuracy.df[i, }\DecValTok{2}\NormalTok{] }\OtherTok{\textless{}{-}} \FunctionTok{confusionMatrix}\NormalTok{(knn.pred, }\FunctionTok{as.factor}\NormalTok{(Test\_labels))}\SpecialCharTok{$}\NormalTok{overall[}\StringTok{"Accuracy"}\NormalTok{]}
\NormalTok{\}}

\NormalTok{accuracy.df}
\end{Highlighting}
\end{Shaded}

\begin{verbatim}
##     k accuracy
## 1   1   0.9630
## 2   2   0.9475
## 3   3   0.9540
## 4   4   0.9470
## 5   5   0.9485
## 6   6   0.9425
## 7   7   0.9460
## 8   8   0.9380
## 9   9   0.9400
## 10 10   0.9370
## 11 11   0.9385
## 12 12   0.9365
## 13 13   0.9390
## 14 14   0.9360
## 15 15   0.9365
## 16 16   0.9325
## 17 17   0.9335
## 18 18   0.9290
## 19 19   0.9290
## 20 20   0.9270
\end{verbatim}

\begin{enumerate}
\def\labelenumi{\arabic{enumi}.}
\setcounter{enumi}{1}
\tightlist
\item
  A choice of K that balances over fitting and ignoring the predictor
  information usually requires testing multiple K values and calculating
  the accuracy of the model on substantial testing data. In this case I
  tested k values 1 through 20 and found the highest accuracy of 96\%
  for K of 1 but to avoid overfitting I would choose a k of 5 which was
  only lower in accuracy by less than 1\%.
\end{enumerate}

\begin{Shaded}
\begin{Highlighting}[]
\FunctionTok{library}\NormalTok{(}\StringTok{"gmodels"}\NormalTok{)}
\end{Highlighting}
\end{Shaded}

\begin{verbatim}
## Warning: package 'gmodels' was built under R version 4.4.3
\end{verbatim}

\begin{Shaded}
\begin{Highlighting}[]
\CommentTok{\# Create the model with the optimized K}
\NormalTok{k1\_Predicted }\OtherTok{\textless{}{-}}\FunctionTok{knn}\NormalTok{(Train\_Predictors, }
\NormalTok{                           Test\_Predictors, }
                           \AttributeTok{cl=}\NormalTok{Train\_labels, }
                           \AttributeTok{k=}\DecValTok{5}\NormalTok{)}

\CommentTok{\# Print the confusion matrix for question 3}
\FunctionTok{CrossTable}\NormalTok{(}\AttributeTok{x=}\NormalTok{Test\_labels,}\AttributeTok{y=}\NormalTok{k1\_Predicted, }\AttributeTok{prop.chisq =} \ConstantTok{FALSE}\NormalTok{)}
\end{Highlighting}
\end{Shaded}

\begin{verbatim}
## 
##  
##    Cell Contents
## |-------------------------|
## |                       N |
## |           N / Row Total |
## |           N / Col Total |
## |         N / Table Total |
## |-------------------------|
## 
##  
## Total Observations in Table:  2000 
## 
##  
##              | k1_Predicted 
##  Test_labels |         0 |         1 | Row Total | 
## -------------|-----------|-----------|-----------|
##            0 |      1798 |         8 |      1806 | 
##              |     0.996 |     0.004 |     0.903 | 
##              |     0.950 |     0.075 |           | 
##              |     0.899 |     0.004 |           | 
## -------------|-----------|-----------|-----------|
##            1 |        95 |        99 |       194 | 
##              |     0.490 |     0.510 |     0.097 | 
##              |     0.050 |     0.925 |           | 
##              |     0.048 |     0.050 |           | 
## -------------|-----------|-----------|-----------|
## Column Total |      1893 |       107 |      2000 | 
##              |     0.947 |     0.053 |           | 
## -------------|-----------|-----------|-----------|
## 
## 
\end{verbatim}

\begin{Shaded}
\begin{Highlighting}[]
\CommentTok{\# Create the KNN model using the previous customer as the predictor and with the new K value of 5}
\NormalTok{Customer\_Test\_labels }\OtherTok{\textless{}{-}}\FunctionTok{knn}\NormalTok{(Train\_Predictors, }
\NormalTok{                           Customer\_Predictors, }
                           \AttributeTok{cl=}\NormalTok{Train\_labels, }
                           \AttributeTok{k=}\DecValTok{5}\NormalTok{)}

\FunctionTok{head}\NormalTok{(Customer\_Test\_labels)}
\end{Highlighting}
\end{Shaded}

\begin{verbatim}
## [1] 0
## Levels: 0
\end{verbatim}

\begin{Shaded}
\begin{Highlighting}[]
\CommentTok{\# Partition the data into 50\% training 30\% validation and 20\% test}
\NormalTok{Train\_index2 }\OtherTok{\textless{}{-}} \FunctionTok{createDataPartition}\NormalTok{(Bank\_normalized}\SpecialCharTok{$}\NormalTok{Personal.Loan, }\AttributeTok{p=}\FloatTok{0.50}\NormalTok{, }\AttributeTok{list=}\ConstantTok{FALSE}\NormalTok{)}

\NormalTok{Train2 }\OtherTok{\textless{}{-}}\NormalTok{ Bank\_normalized[Train\_index2,]}
\NormalTok{Place\_holder }\OtherTok{\textless{}{-}}\NormalTok{ Bank\_normalized[}\SpecialCharTok{{-}}\NormalTok{Train\_index2,]}

\NormalTok{Train\_index3 }\OtherTok{\textless{}{-}} \FunctionTok{createDataPartition}\NormalTok{(Place\_holder}\SpecialCharTok{$}\NormalTok{Personal.Loan, }\AttributeTok{p=}\FloatTok{0.6}\NormalTok{, }\AttributeTok{list=}\ConstantTok{FALSE}\NormalTok{)}

\NormalTok{Validation2 }\OtherTok{\textless{}{-}}\NormalTok{ Place\_holder[Train\_index3,]}
\NormalTok{Test2 }\OtherTok{\textless{}{-}}\NormalTok{ Place\_holder[}\SpecialCharTok{{-}}\NormalTok{Train\_index3,]}

\CommentTok{\# Select the correct columns for training and testing variables}
\NormalTok{Train\_Predictors2 }\OtherTok{\textless{}{-}}\NormalTok{ Train2[,}\SpecialCharTok{{-}}\FunctionTok{c}\NormalTok{(}\DecValTok{4}\NormalTok{, }\DecValTok{8}\NormalTok{, }\DecValTok{12}\NormalTok{)]}
\NormalTok{Test\_Predictors2 }\OtherTok{\textless{}{-}}\NormalTok{ Test2[,}\SpecialCharTok{{-}}\FunctionTok{c}\NormalTok{(}\DecValTok{4}\NormalTok{, }\DecValTok{8}\NormalTok{, }\DecValTok{12}\NormalTok{)]}
\NormalTok{Validation\_Predictors2 }\OtherTok{\textless{}{-}}\NormalTok{ Validation2[,}\SpecialCharTok{{-}}\FunctionTok{c}\NormalTok{(}\DecValTok{4}\NormalTok{, }\DecValTok{8}\NormalTok{, }\DecValTok{12}\NormalTok{)]}

\NormalTok{Train\_labels2 }\OtherTok{\textless{}{-}}\NormalTok{ Train2[,}\DecValTok{12}\NormalTok{] }
\NormalTok{Test\_labels2  }\OtherTok{\textless{}{-}}\NormalTok{ Test2[,}\DecValTok{12}\NormalTok{] }
\NormalTok{Validation\_labels2  }\OtherTok{\textless{}{-}}\NormalTok{ Validation2[,}\DecValTok{12}\NormalTok{] }
\end{Highlighting}
\end{Shaded}

\begin{Shaded}
\begin{Highlighting}[]
\CommentTok{\# Testing data performance}
\NormalTok{Q5TestSet }\OtherTok{\textless{}{-}}\FunctionTok{knn}\NormalTok{(Train\_Predictors2, }
\NormalTok{                           Test\_Predictors2, }
                           \AttributeTok{cl=}\NormalTok{Train\_labels2, }
                           \AttributeTok{k=}\DecValTok{5}\NormalTok{)}

\FunctionTok{CrossTable}\NormalTok{(}\AttributeTok{x=}\NormalTok{Test\_labels2,}\AttributeTok{y=}\NormalTok{Q5TestSet, }\AttributeTok{prop.chisq =} \ConstantTok{FALSE}\NormalTok{)}
\end{Highlighting}
\end{Shaded}

\begin{verbatim}
## 
##  
##    Cell Contents
## |-------------------------|
## |                       N |
## |           N / Row Total |
## |           N / Col Total |
## |         N / Table Total |
## |-------------------------|
## 
##  
## Total Observations in Table:  1000 
## 
##  
##              | Q5TestSet 
## Test_labels2 |         0 |         1 | Row Total | 
## -------------|-----------|-----------|-----------|
##            0 |       901 |         3 |       904 | 
##              |     0.997 |     0.003 |     0.904 | 
##              |     0.949 |     0.059 |           | 
##              |     0.901 |     0.003 |           | 
## -------------|-----------|-----------|-----------|
##            1 |        48 |        48 |        96 | 
##              |     0.500 |     0.500 |     0.096 | 
##              |     0.051 |     0.941 |           | 
##              |     0.048 |     0.048 |           | 
## -------------|-----------|-----------|-----------|
## Column Total |       949 |        51 |      1000 | 
##              |     0.949 |     0.051 |           | 
## -------------|-----------|-----------|-----------|
## 
## 
\end{verbatim}

\begin{Shaded}
\begin{Highlighting}[]
\CommentTok{\# Validation data performance}
\NormalTok{Q5ValidationSet }\OtherTok{\textless{}{-}}\FunctionTok{knn}\NormalTok{(Train\_Predictors2, }
\NormalTok{                           Validation\_Predictors2, }
                           \AttributeTok{cl=}\NormalTok{Train\_labels2, }
                           \AttributeTok{k=}\DecValTok{5}\NormalTok{)}

\FunctionTok{CrossTable}\NormalTok{(}\AttributeTok{x=}\NormalTok{Validation\_labels2,}\AttributeTok{y=}\NormalTok{Q5ValidationSet, }\AttributeTok{prop.chisq =} \ConstantTok{FALSE}\NormalTok{)}
\end{Highlighting}
\end{Shaded}

\begin{verbatim}
## 
##  
##    Cell Contents
## |-------------------------|
## |                       N |
## |           N / Row Total |
## |           N / Col Total |
## |         N / Table Total |
## |-------------------------|
## 
##  
## Total Observations in Table:  1500 
## 
##  
##                    | Q5ValidationSet 
## Validation_labels2 |         0 |         1 | Row Total | 
## -------------------|-----------|-----------|-----------|
##                  0 |      1350 |         4 |      1354 | 
##                    |     0.997 |     0.003 |     0.903 | 
##                    |     0.943 |     0.058 |           | 
##                    |     0.900 |     0.003 |           | 
## -------------------|-----------|-----------|-----------|
##                  1 |        81 |        65 |       146 | 
##                    |     0.555 |     0.445 |     0.097 | 
##                    |     0.057 |     0.942 |           | 
##                    |     0.054 |     0.043 |           | 
## -------------------|-----------|-----------|-----------|
##       Column Total |      1431 |        69 |      1500 | 
##                    |     0.954 |     0.046 |           | 
## -------------------|-----------|-----------|-----------|
## 
## 
\end{verbatim}

\begin{Shaded}
\begin{Highlighting}[]
\CommentTok{\# Training data performance}
\NormalTok{Q5TrainSet }\OtherTok{\textless{}{-}}\FunctionTok{knn}\NormalTok{(Train\_Predictors2, }
\NormalTok{                           Train\_Predictors2, }
                           \AttributeTok{cl=}\NormalTok{Train\_labels2, }
                           \AttributeTok{k=}\DecValTok{5}\NormalTok{)}

\FunctionTok{CrossTable}\NormalTok{(}\AttributeTok{x=}\NormalTok{Train\_labels2,}\AttributeTok{y=}\NormalTok{Q5TrainSet, }\AttributeTok{prop.chisq =} \ConstantTok{FALSE}\NormalTok{)}
\end{Highlighting}
\end{Shaded}

\begin{verbatim}
## 
##  
##    Cell Contents
## |-------------------------|
## |                       N |
## |           N / Row Total |
## |           N / Col Total |
## |         N / Table Total |
## |-------------------------|
## 
##  
## Total Observations in Table:  2500 
## 
##  
##               | Q5TrainSet 
## Train_labels2 |         0 |         1 | Row Total | 
## --------------|-----------|-----------|-----------|
##             0 |      2256 |         6 |      2262 | 
##               |     0.997 |     0.003 |     0.905 | 
##               |     0.969 |     0.035 |           | 
##               |     0.902 |     0.002 |           | 
## --------------|-----------|-----------|-----------|
##             1 |        72 |       166 |       238 | 
##               |     0.303 |     0.697 |     0.095 | 
##               |     0.031 |     0.965 |           | 
##               |     0.029 |     0.066 |           | 
## --------------|-----------|-----------|-----------|
##  Column Total |      2328 |       172 |      2500 | 
##               |     0.931 |     0.069 |           | 
## --------------|-----------|-----------|-----------|
## 
## 
\end{verbatim}

\begin{enumerate}
\def\labelenumi{\arabic{enumi}.}
\setcounter{enumi}{4}
\tightlist
\item
  The training set performed the best, which makes sense as the model
  has already utilized that data for the creation of the model. The
  validation set performed the next best which I would assume is due to
  it being a larger data set of 1500 data points compared to the test
  set of 1000 but they were mostly similar due to them both being unseen
  data to the model.
\end{enumerate}

\end{document}
