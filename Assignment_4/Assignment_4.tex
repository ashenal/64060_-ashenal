% Options for packages loaded elsewhere
\PassOptionsToPackage{unicode}{hyperref}
\PassOptionsToPackage{hyphens}{url}
%
\documentclass[
]{article}
\usepackage{amsmath,amssymb}
\usepackage{iftex}
\ifPDFTeX
  \usepackage[T1]{fontenc}
  \usepackage[utf8]{inputenc}
  \usepackage{textcomp} % provide euro and other symbols
\else % if luatex or xetex
  \usepackage{unicode-math} % this also loads fontspec
  \defaultfontfeatures{Scale=MatchLowercase}
  \defaultfontfeatures[\rmfamily]{Ligatures=TeX,Scale=1}
\fi
\usepackage{lmodern}
\ifPDFTeX\else
  % xetex/luatex font selection
\fi
% Use upquote if available, for straight quotes in verbatim environments
\IfFileExists{upquote.sty}{\usepackage{upquote}}{}
\IfFileExists{microtype.sty}{% use microtype if available
  \usepackage[]{microtype}
  \UseMicrotypeSet[protrusion]{basicmath} % disable protrusion for tt fonts
}{}
\makeatletter
\@ifundefined{KOMAClassName}{% if non-KOMA class
  \IfFileExists{parskip.sty}{%
    \usepackage{parskip}
  }{% else
    \setlength{\parindent}{0pt}
    \setlength{\parskip}{6pt plus 2pt minus 1pt}}
}{% if KOMA class
  \KOMAoptions{parskip=half}}
\makeatother
\usepackage{xcolor}
\usepackage[margin=1in]{geometry}
\usepackage{color}
\usepackage{fancyvrb}
\newcommand{\VerbBar}{|}
\newcommand{\VERB}{\Verb[commandchars=\\\{\}]}
\DefineVerbatimEnvironment{Highlighting}{Verbatim}{commandchars=\\\{\}}
% Add ',fontsize=\small' for more characters per line
\usepackage{framed}
\definecolor{shadecolor}{RGB}{248,248,248}
\newenvironment{Shaded}{\begin{snugshade}}{\end{snugshade}}
\newcommand{\AlertTok}[1]{\textcolor[rgb]{0.94,0.16,0.16}{#1}}
\newcommand{\AnnotationTok}[1]{\textcolor[rgb]{0.56,0.35,0.01}{\textbf{\textit{#1}}}}
\newcommand{\AttributeTok}[1]{\textcolor[rgb]{0.13,0.29,0.53}{#1}}
\newcommand{\BaseNTok}[1]{\textcolor[rgb]{0.00,0.00,0.81}{#1}}
\newcommand{\BuiltInTok}[1]{#1}
\newcommand{\CharTok}[1]{\textcolor[rgb]{0.31,0.60,0.02}{#1}}
\newcommand{\CommentTok}[1]{\textcolor[rgb]{0.56,0.35,0.01}{\textit{#1}}}
\newcommand{\CommentVarTok}[1]{\textcolor[rgb]{0.56,0.35,0.01}{\textbf{\textit{#1}}}}
\newcommand{\ConstantTok}[1]{\textcolor[rgb]{0.56,0.35,0.01}{#1}}
\newcommand{\ControlFlowTok}[1]{\textcolor[rgb]{0.13,0.29,0.53}{\textbf{#1}}}
\newcommand{\DataTypeTok}[1]{\textcolor[rgb]{0.13,0.29,0.53}{#1}}
\newcommand{\DecValTok}[1]{\textcolor[rgb]{0.00,0.00,0.81}{#1}}
\newcommand{\DocumentationTok}[1]{\textcolor[rgb]{0.56,0.35,0.01}{\textbf{\textit{#1}}}}
\newcommand{\ErrorTok}[1]{\textcolor[rgb]{0.64,0.00,0.00}{\textbf{#1}}}
\newcommand{\ExtensionTok}[1]{#1}
\newcommand{\FloatTok}[1]{\textcolor[rgb]{0.00,0.00,0.81}{#1}}
\newcommand{\FunctionTok}[1]{\textcolor[rgb]{0.13,0.29,0.53}{\textbf{#1}}}
\newcommand{\ImportTok}[1]{#1}
\newcommand{\InformationTok}[1]{\textcolor[rgb]{0.56,0.35,0.01}{\textbf{\textit{#1}}}}
\newcommand{\KeywordTok}[1]{\textcolor[rgb]{0.13,0.29,0.53}{\textbf{#1}}}
\newcommand{\NormalTok}[1]{#1}
\newcommand{\OperatorTok}[1]{\textcolor[rgb]{0.81,0.36,0.00}{\textbf{#1}}}
\newcommand{\OtherTok}[1]{\textcolor[rgb]{0.56,0.35,0.01}{#1}}
\newcommand{\PreprocessorTok}[1]{\textcolor[rgb]{0.56,0.35,0.01}{\textit{#1}}}
\newcommand{\RegionMarkerTok}[1]{#1}
\newcommand{\SpecialCharTok}[1]{\textcolor[rgb]{0.81,0.36,0.00}{\textbf{#1}}}
\newcommand{\SpecialStringTok}[1]{\textcolor[rgb]{0.31,0.60,0.02}{#1}}
\newcommand{\StringTok}[1]{\textcolor[rgb]{0.31,0.60,0.02}{#1}}
\newcommand{\VariableTok}[1]{\textcolor[rgb]{0.00,0.00,0.00}{#1}}
\newcommand{\VerbatimStringTok}[1]{\textcolor[rgb]{0.31,0.60,0.02}{#1}}
\newcommand{\WarningTok}[1]{\textcolor[rgb]{0.56,0.35,0.01}{\textbf{\textit{#1}}}}
\usepackage{graphicx}
\makeatletter
\def\maxwidth{\ifdim\Gin@nat@width>\linewidth\linewidth\else\Gin@nat@width\fi}
\def\maxheight{\ifdim\Gin@nat@height>\textheight\textheight\else\Gin@nat@height\fi}
\makeatother
% Scale images if necessary, so that they will not overflow the page
% margins by default, and it is still possible to overwrite the defaults
% using explicit options in \includegraphics[width, height, ...]{}
\setkeys{Gin}{width=\maxwidth,height=\maxheight,keepaspectratio}
% Set default figure placement to htbp
\makeatletter
\def\fps@figure{htbp}
\makeatother
\setlength{\emergencystretch}{3em} % prevent overfull lines
\providecommand{\tightlist}{%
  \setlength{\itemsep}{0pt}\setlength{\parskip}{0pt}}
\setcounter{secnumdepth}{-\maxdimen} % remove section numbering
\ifLuaTeX
  \usepackage{selnolig}  % disable illegal ligatures
\fi
\usepackage{bookmark}
\IfFileExists{xurl.sty}{\usepackage{xurl}}{} % add URL line breaks if available
\urlstyle{same}
\hypersetup{
  pdftitle={Assignment\_4},
  pdfauthor={Andrew Shenal},
  hidelinks,
  pdfcreator={LaTeX via pandoc}}

\title{Assignment\_4}
\author{Andrew Shenal}
\date{2025-10-21}

\begin{document}
\maketitle

\begin{Shaded}
\begin{Highlighting}[]
\FunctionTok{library}\NormalTok{(factoextra)}
\FunctionTok{library}\NormalTok{(tidyverse)}
\FunctionTok{library}\NormalTok{(flexclust)}

\NormalTok{df1 }\OtherTok{\textless{}{-}} \FunctionTok{read\_csv}\NormalTok{(}\StringTok{"./Pharmaceuticals.csv"}\NormalTok{) }\CommentTok{\# Loads in csv file}
\NormalTok{df }\OtherTok{\textless{}{-}}\NormalTok{ df1[,}\FunctionTok{c}\NormalTok{(}\DecValTok{3}\SpecialCharTok{:}\DecValTok{11}\NormalTok{)] }\CommentTok{\# Selects only columns 3 through 11 (the numerical variables) to use in the analysis}
\NormalTok{df }\OtherTok{\textless{}{-}} \FunctionTok{as.data.frame}\NormalTok{(df)}
\FunctionTok{set.seed}\NormalTok{(}\DecValTok{100}\NormalTok{)}
\end{Highlighting}
\end{Shaded}

\begin{enumerate}
\def\labelenumi{\alph{enumi}.}
\tightlist
\item
  Use only the numerical variables (1 to 9) to cluster the 21 firms.
  Justify the various choices made in conducting the cluster analysis,
  such as weights for different variables, the specific clustering
  algorithm(s) used, the number of clusters formed, and so on.
\end{enumerate}

\begin{Shaded}
\begin{Highlighting}[]
\NormalTok{df }\OtherTok{\textless{}{-}} \FunctionTok{scale}\NormalTok{(df) }\CommentTok{\# Normalize the data}
\FunctionTok{summary}\NormalTok{(df)}
\end{Highlighting}
\end{Shaded}

\begin{verbatim}
##    Market_Cap           Beta            PE_Ratio            ROE         
##  Min.   :-0.9768   Min.   :-1.3466   Min.   :-1.3404   Min.   :-1.4515  
##  1st Qu.:-0.8763   1st Qu.:-0.6844   1st Qu.:-0.4023   1st Qu.:-0.7223  
##  Median :-0.1614   Median :-0.2560   Median :-0.2429   Median :-0.2118  
##  Mean   : 0.0000   Mean   : 0.0000   Mean   : 0.0000   Mean   : 0.0000  
##  3rd Qu.: 0.2762   3rd Qu.: 0.4841   3rd Qu.: 0.1495   3rd Qu.: 0.3450  
##  Max.   : 2.4200   Max.   : 2.2758   Max.   : 3.4971   Max.   : 2.4597  
##       ROA          Asset_Turnover       Leverage          Rev_Growth     
##  Min.   :-1.7128   Min.   :-1.8451   Min.   :-0.74966   Min.   :-1.4971  
##  1st Qu.:-0.9047   1st Qu.:-0.4613   1st Qu.:-0.54487   1st Qu.:-0.6328  
##  Median : 0.1289   Median :-0.4613   Median :-0.31449   Median :-0.3621  
##  Mean   : 0.0000   Mean   : 0.0000   Mean   : 0.00000   Mean   : 0.0000  
##  3rd Qu.: 0.8430   3rd Qu.: 0.9225   3rd Qu.: 0.01828   3rd Qu.: 0.7693  
##  Max.   : 1.8389   Max.   : 1.8451   Max.   : 3.74280   Max.   : 1.8862  
##  Net_Profit_Margin 
##  Min.   :-1.99560  
##  1st Qu.:-0.68504  
##  Median : 0.06168  
##  Mean   : 0.00000  
##  3rd Qu.: 0.82364  
##  Max.   : 1.49416
\end{verbatim}

\begin{Shaded}
\begin{Highlighting}[]
\FunctionTok{fviz\_nbclust}\NormalTok{(df, kmeans, }\AttributeTok{method =} \StringTok{"silhouette"}\NormalTok{)}
\end{Highlighting}
\end{Shaded}

\includegraphics{Assignment_4_files/figure-latex/unnamed-chunk-2-1.pdf}
The average silhouette method shows five as the best number of clusters.
Additionally, the data has a small sample set so I chose kmedians in
order to limit any large impact from a single data point.

\begin{Shaded}
\begin{Highlighting}[]
\NormalTok{k4 }\OtherTok{=} \FunctionTok{kcca}\NormalTok{(df, }\AttributeTok{k=}\DecValTok{5}\NormalTok{, }\FunctionTok{kccaFamily}\NormalTok{(}\StringTok{"kmedians"}\NormalTok{))}
\NormalTok{k4 }\CommentTok{\#prints cluster sizes}
\end{Highlighting}
\end{Shaded}

\begin{verbatim}
## kcca object of family 'kmedians' 
## 
## call:
## kcca(x = df, k = 5, family = kccaFamily("kmedians"))
## 
## cluster sizes:
## 
##  1  2  3  4  5 
##  2  2  2 10  5
\end{verbatim}

\begin{Shaded}
\begin{Highlighting}[]
\NormalTok{clusters\_index }\OtherTok{\textless{}{-}} \FunctionTok{predict}\NormalTok{(k4)}
\FunctionTok{dist}\NormalTok{(k4}\SpecialCharTok{@}\NormalTok{centers)}
\end{Highlighting}
\end{Shaded}

\begin{verbatim}
##          1        2        3        4
## 2 2.517329                           
## 3 3.224568 3.952024                  
## 4 3.622472 3.272419 5.228926         
## 5 2.109198 2.313205 2.984957 2.925318
\end{verbatim}

\begin{Shaded}
\begin{Highlighting}[]
\FunctionTok{image}\NormalTok{(k4)}
\FunctionTok{points}\NormalTok{(df, }\AttributeTok{col=}\NormalTok{clusters\_index, }\AttributeTok{pch=}\DecValTok{19}\NormalTok{, }\AttributeTok{cex=}\FloatTok{0.6}\NormalTok{)}
\end{Highlighting}
\end{Shaded}

\includegraphics{Assignment_4_files/figure-latex/unnamed-chunk-3-1.pdf}

\begin{Shaded}
\begin{Highlighting}[]
\NormalTok{df}\SpecialCharTok{$}\NormalTok{cluster }\OtherTok{\textless{}{-}} \FunctionTok{predict}\NormalTok{(k4)}
\end{Highlighting}
\end{Shaded}

\begin{verbatim}
## Warning in df$cluster <- predict(k4): Coercing LHS to a list
\end{verbatim}

\begin{Shaded}
\begin{Highlighting}[]
\NormalTok{df1 }\OtherTok{\textless{}{-}} \FunctionTok{as.data.frame}\NormalTok{(df1)}
\NormalTok{df1}\SpecialCharTok{$}\NormalTok{cluster }\OtherTok{\textless{}{-}}\NormalTok{ clusters\_index}
\end{Highlighting}
\end{Shaded}

\begin{enumerate}
\def\labelenumi{\alph{enumi}.}
\setcounter{enumi}{1}
\tightlist
\item
  Interpret the clusters with respect to the numerical variables used in
  forming the clusters.
\end{enumerate}

Cluster 1: Shows companies that have stable growth and steady return on
investment Cluster 2: Companies that have a larger share of the market
cap with still consistent ROA and profit. Cluster 3: Companies that have
strong profitability with high returns (ROI ROA). Cluster 4: These
companies are smaller with greater variability in their returns. With
higher beta and debt they are riskier overall. Cluster 5: These are very
large companies with steady revenue and limited volatility (beta) and
high asset turnover that suggests efficiency.

\begin{enumerate}
\def\labelenumi{\alph{enumi}.}
\setcounter{enumi}{2}
\tightlist
\item
  Is there a pattern in the clusters with respect to the numerical
  variables (10 to 12)? (those not used in forming the clusters)
\end{enumerate}

Yes, the most stable and large companies represented in clusters 1, 2,
and 5 had had recommendations for Hold while the faster growing
companies or more volatile companies in cluster 3 and 4 had
recommendations for action - moderate buy or sell. I also noticed that
the majority of locations outside the U.S. were represented in clusters
3 and 4.

\begin{enumerate}
\def\labelenumi{\alph{enumi}.}
\setcounter{enumi}{3}
\tightlist
\item
  Provide an appropriate name for each cluster using any or all of the
  variables in the dataset.
\end{enumerate}

\begin{enumerate}
\def\labelenumi{\arabic{enumi}.}
\tightlist
\item
  Stable average performing companies
\item
  Large stable low risk companies
\item
  High performing growing companies
\item
  Volatile Performers
\item
  Large financially strong companies
\end{enumerate}

\end{document}
