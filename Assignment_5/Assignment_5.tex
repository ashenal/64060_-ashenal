% Options for packages loaded elsewhere
\PassOptionsToPackage{unicode}{hyperref}
\PassOptionsToPackage{hyphens}{url}
%
\documentclass[
]{article}
\usepackage{amsmath,amssymb}
\usepackage{iftex}
\ifPDFTeX
  \usepackage[T1]{fontenc}
  \usepackage[utf8]{inputenc}
  \usepackage{textcomp} % provide euro and other symbols
\else % if luatex or xetex
  \usepackage{unicode-math} % this also loads fontspec
  \defaultfontfeatures{Scale=MatchLowercase}
  \defaultfontfeatures[\rmfamily]{Ligatures=TeX,Scale=1}
\fi
\usepackage{lmodern}
\ifPDFTeX\else
  % xetex/luatex font selection
\fi
% Use upquote if available, for straight quotes in verbatim environments
\IfFileExists{upquote.sty}{\usepackage{upquote}}{}
\IfFileExists{microtype.sty}{% use microtype if available
  \usepackage[]{microtype}
  \UseMicrotypeSet[protrusion]{basicmath} % disable protrusion for tt fonts
}{}
\makeatletter
\@ifundefined{KOMAClassName}{% if non-KOMA class
  \IfFileExists{parskip.sty}{%
    \usepackage{parskip}
  }{% else
    \setlength{\parindent}{0pt}
    \setlength{\parskip}{6pt plus 2pt minus 1pt}}
}{% if KOMA class
  \KOMAoptions{parskip=half}}
\makeatother
\usepackage{xcolor}
\usepackage[margin=1in]{geometry}
\usepackage{color}
\usepackage{fancyvrb}
\newcommand{\VerbBar}{|}
\newcommand{\VERB}{\Verb[commandchars=\\\{\}]}
\DefineVerbatimEnvironment{Highlighting}{Verbatim}{commandchars=\\\{\}}
% Add ',fontsize=\small' for more characters per line
\usepackage{framed}
\definecolor{shadecolor}{RGB}{248,248,248}
\newenvironment{Shaded}{\begin{snugshade}}{\end{snugshade}}
\newcommand{\AlertTok}[1]{\textcolor[rgb]{0.94,0.16,0.16}{#1}}
\newcommand{\AnnotationTok}[1]{\textcolor[rgb]{0.56,0.35,0.01}{\textbf{\textit{#1}}}}
\newcommand{\AttributeTok}[1]{\textcolor[rgb]{0.13,0.29,0.53}{#1}}
\newcommand{\BaseNTok}[1]{\textcolor[rgb]{0.00,0.00,0.81}{#1}}
\newcommand{\BuiltInTok}[1]{#1}
\newcommand{\CharTok}[1]{\textcolor[rgb]{0.31,0.60,0.02}{#1}}
\newcommand{\CommentTok}[1]{\textcolor[rgb]{0.56,0.35,0.01}{\textit{#1}}}
\newcommand{\CommentVarTok}[1]{\textcolor[rgb]{0.56,0.35,0.01}{\textbf{\textit{#1}}}}
\newcommand{\ConstantTok}[1]{\textcolor[rgb]{0.56,0.35,0.01}{#1}}
\newcommand{\ControlFlowTok}[1]{\textcolor[rgb]{0.13,0.29,0.53}{\textbf{#1}}}
\newcommand{\DataTypeTok}[1]{\textcolor[rgb]{0.13,0.29,0.53}{#1}}
\newcommand{\DecValTok}[1]{\textcolor[rgb]{0.00,0.00,0.81}{#1}}
\newcommand{\DocumentationTok}[1]{\textcolor[rgb]{0.56,0.35,0.01}{\textbf{\textit{#1}}}}
\newcommand{\ErrorTok}[1]{\textcolor[rgb]{0.64,0.00,0.00}{\textbf{#1}}}
\newcommand{\ExtensionTok}[1]{#1}
\newcommand{\FloatTok}[1]{\textcolor[rgb]{0.00,0.00,0.81}{#1}}
\newcommand{\FunctionTok}[1]{\textcolor[rgb]{0.13,0.29,0.53}{\textbf{#1}}}
\newcommand{\ImportTok}[1]{#1}
\newcommand{\InformationTok}[1]{\textcolor[rgb]{0.56,0.35,0.01}{\textbf{\textit{#1}}}}
\newcommand{\KeywordTok}[1]{\textcolor[rgb]{0.13,0.29,0.53}{\textbf{#1}}}
\newcommand{\NormalTok}[1]{#1}
\newcommand{\OperatorTok}[1]{\textcolor[rgb]{0.81,0.36,0.00}{\textbf{#1}}}
\newcommand{\OtherTok}[1]{\textcolor[rgb]{0.56,0.35,0.01}{#1}}
\newcommand{\PreprocessorTok}[1]{\textcolor[rgb]{0.56,0.35,0.01}{\textit{#1}}}
\newcommand{\RegionMarkerTok}[1]{#1}
\newcommand{\SpecialCharTok}[1]{\textcolor[rgb]{0.81,0.36,0.00}{\textbf{#1}}}
\newcommand{\SpecialStringTok}[1]{\textcolor[rgb]{0.31,0.60,0.02}{#1}}
\newcommand{\StringTok}[1]{\textcolor[rgb]{0.31,0.60,0.02}{#1}}
\newcommand{\VariableTok}[1]{\textcolor[rgb]{0.00,0.00,0.00}{#1}}
\newcommand{\VerbatimStringTok}[1]{\textcolor[rgb]{0.31,0.60,0.02}{#1}}
\newcommand{\WarningTok}[1]{\textcolor[rgb]{0.56,0.35,0.01}{\textbf{\textit{#1}}}}
\usepackage{graphicx}
\makeatletter
\def\maxwidth{\ifdim\Gin@nat@width>\linewidth\linewidth\else\Gin@nat@width\fi}
\def\maxheight{\ifdim\Gin@nat@height>\textheight\textheight\else\Gin@nat@height\fi}
\makeatother
% Scale images if necessary, so that they will not overflow the page
% margins by default, and it is still possible to overwrite the defaults
% using explicit options in \includegraphics[width, height, ...]{}
\setkeys{Gin}{width=\maxwidth,height=\maxheight,keepaspectratio}
% Set default figure placement to htbp
\makeatletter
\def\fps@figure{htbp}
\makeatother
\setlength{\emergencystretch}{3em} % prevent overfull lines
\providecommand{\tightlist}{%
  \setlength{\itemsep}{0pt}\setlength{\parskip}{0pt}}
\setcounter{secnumdepth}{-\maxdimen} % remove section numbering
\ifLuaTeX
  \usepackage{selnolig}  % disable illegal ligatures
\fi
\usepackage{bookmark}
\IfFileExists{xurl.sty}{\usepackage{xurl}}{} % add URL line breaks if available
\urlstyle{same}
\hypersetup{
  pdftitle={Assignment\_5},
  pdfauthor={Andrew Shenal},
  hidelinks,
  pdfcreator={LaTeX via pandoc}}

\title{Assignment\_5}
\author{Andrew Shenal}
\date{2025-11-22}

\begin{document}
\maketitle

\begin{Shaded}
\begin{Highlighting}[]
\CommentTok{\# Package Set{-}Up}
\FunctionTok{library}\NormalTok{(stats)}
\FunctionTok{library}\NormalTok{(cluster)}
\FunctionTok{library}\NormalTok{(mclust)}
\end{Highlighting}
\end{Shaded}

\begin{verbatim}
## Warning: package 'mclust' was built under R version 4.4.3
\end{verbatim}

\begin{verbatim}
## Package 'mclust' version 6.1.2
## Type 'citation("mclust")' for citing this R package in publications.
\end{verbatim}

\begin{Shaded}
\begin{Highlighting}[]
\FunctionTok{set.seed}\NormalTok{(}\DecValTok{111}\NormalTok{)}

\CommentTok{\# Load data}
\NormalTok{df }\OtherTok{\textless{}{-}} \FunctionTok{read.csv}\NormalTok{(}\StringTok{"./Cereals.csv"}\NormalTok{)}

\CommentTok{\# Make the cereal name the row names and remove the non{-}numerical columns for normalization}
\FunctionTok{row.names}\NormalTok{(df) }\OtherTok{\textless{}{-}}\NormalTok{ df[,}\DecValTok{1}\NormalTok{]}
\NormalTok{df }\OtherTok{\textless{}{-}}\NormalTok{ df[,}\SpecialCharTok{{-}}\NormalTok{(}\DecValTok{1}\SpecialCharTok{:}\DecValTok{3}\NormalTok{)]}

\CommentTok{\# Remove missing values and Scale data}
\NormalTok{df }\OtherTok{\textless{}{-}} \FunctionTok{na.omit}\NormalTok{(df)}
\NormalTok{df }\OtherTok{\textless{}{-}} \FunctionTok{scale}\NormalTok{(df)}
\end{Highlighting}
\end{Shaded}

\begin{enumerate}
\def\labelenumi{\arabic{enumi}.}
\tightlist
\item
  Apply hierarchical clustering to the data using Euclidean distance to
  the normalized measurements. Use Agnes to compare the clustering from
  single linkage, complete linkage, average linkage, and Ward. Choose
  the best method.
\end{enumerate}

\begin{Shaded}
\begin{Highlighting}[]
\CommentTok{\# Compute euclidean distance on the normalized data}
\NormalTok{d }\OtherTok{\textless{}{-}} \FunctionTok{dist}\NormalTok{(df, }\AttributeTok{method =} \StringTok{"euclidean"}\NormalTok{)}

\CommentTok{\# Use agnes to apply the 4 clustering structures}
\NormalTok{hc\_single }\OtherTok{\textless{}{-}} \FunctionTok{agnes}\NormalTok{(d, }\AttributeTok{method =} \StringTok{"single"}\NormalTok{)}
\NormalTok{hc\_complete }\OtherTok{\textless{}{-}} \FunctionTok{agnes}\NormalTok{(d, }\AttributeTok{method =} \StringTok{"complete"}\NormalTok{)}
\NormalTok{hc\_average }\OtherTok{\textless{}{-}} \FunctionTok{agnes}\NormalTok{(d, }\AttributeTok{method =} \StringTok{"average"}\NormalTok{)}
\NormalTok{hc\_ward }\OtherTok{\textless{}{-}} \FunctionTok{agnes}\NormalTok{(df, }\AttributeTok{method =} \StringTok{"ward"}\NormalTok{)}

\CommentTok{\# Compare Agglomerative coefficients}
\FunctionTok{print}\NormalTok{(hc\_single}\SpecialCharTok{$}\NormalTok{ac) }\CommentTok{\# 0.6067859}
\end{Highlighting}
\end{Shaded}

\begin{verbatim}
## [1] 0.6067859
\end{verbatim}

\begin{Shaded}
\begin{Highlighting}[]
\FunctionTok{print}\NormalTok{(hc\_complete}\SpecialCharTok{$}\NormalTok{ac) }\CommentTok{\# 0.8353712}
\end{Highlighting}
\end{Shaded}

\begin{verbatim}
## [1] 0.8353712
\end{verbatim}

\begin{Shaded}
\begin{Highlighting}[]
\FunctionTok{print}\NormalTok{(hc\_average}\SpecialCharTok{$}\NormalTok{ac) }\CommentTok{\# 0.7766075}
\end{Highlighting}
\end{Shaded}

\begin{verbatim}
## [1] 0.7766075
\end{verbatim}

\begin{Shaded}
\begin{Highlighting}[]
\FunctionTok{print}\NormalTok{(hc\_ward}\SpecialCharTok{$}\NormalTok{ac) }\CommentTok{\# 0.9046042 {-} Highest and best option}
\end{Highlighting}
\end{Shaded}

\begin{verbatim}
## [1] 0.9046042
\end{verbatim}

\begin{Shaded}
\begin{Highlighting}[]
\FunctionTok{pltree}\NormalTok{(hc\_ward, }\AttributeTok{cex =} \FloatTok{0.6}\NormalTok{, }\AttributeTok{hang =} \SpecialCharTok{{-}}\DecValTok{1}\NormalTok{, }\AttributeTok{main =} \StringTok{"Dendrogram of agnes"}\NormalTok{)}
\FunctionTok{rect.hclust}\NormalTok{(hc\_ward, }\AttributeTok{k =} \DecValTok{5}\NormalTok{, }\AttributeTok{border =} \DecValTok{1}\SpecialCharTok{:}\DecValTok{4}\NormalTok{)}
\end{Highlighting}
\end{Shaded}

\includegraphics{Assignment_5_files/figure-latex/unnamed-chunk-2-1.pdf}
- I would choose around 5 clusters. A larger number of clusters would
have weaker differences and reduce the ability to identify differences
while a smaller number would bring too many distinct clusters together.
At around 5 the clusters have enough distance between them to show that
they are different with clear identifiable separation.

Comment on the structure of the clusters and on their stability. Hint:
To check stability, partition the data and see how well clusters formed
based on one part apply to the other part. To do this: ● Cluster
partition A ● Use the cluster centroids from A to assign each record in
partition B (each record is assigned to the cluster with the closest
centroid). ● Assess how consistent the cluster assignments are compared
to the assignments based on all the data.

\begin{Shaded}
\begin{Highlighting}[]
\FunctionTok{library}\NormalTok{(caret)}
\end{Highlighting}
\end{Shaded}

\begin{verbatim}
## Warning: package 'caret' was built under R version 4.4.3
\end{verbatim}

\begin{verbatim}
## Loading required package: ggplot2
\end{verbatim}

\begin{verbatim}
## Warning: package 'ggplot2' was built under R version 4.4.3
\end{verbatim}

\begin{verbatim}
## Loading required package: lattice
\end{verbatim}

\begin{Shaded}
\begin{Highlighting}[]
\FunctionTok{library}\NormalTok{(mclust)}

\FunctionTok{set.seed}\NormalTok{(}\DecValTok{111}\NormalTok{)}

\CommentTok{\# split data}
\NormalTok{idx }\OtherTok{\textless{}{-}} \FunctionTok{createDataPartition}\NormalTok{(}\DecValTok{1}\SpecialCharTok{:}\FunctionTok{nrow}\NormalTok{(df), }\AttributeTok{p =} \FloatTok{0.7}\NormalTok{, }\AttributeTok{list =} \ConstantTok{FALSE}\NormalTok{)}
\NormalTok{train }\OtherTok{\textless{}{-}}\NormalTok{ df[idx, ]}
\NormalTok{test }\OtherTok{\textless{}{-}}\NormalTok{ df[}\SpecialCharTok{{-}}\NormalTok{idx, ]}

\CommentTok{\# hierarchical clustering on training set}
\NormalTok{d\_train }\OtherTok{\textless{}{-}} \FunctionTok{dist}\NormalTok{(train)}
\NormalTok{hc\_train }\OtherTok{\textless{}{-}} \FunctionTok{agnes}\NormalTok{(d\_train, }\AttributeTok{method =} \StringTok{"ward"}\NormalTok{)}
\NormalTok{cl\_train }\OtherTok{\textless{}{-}} \FunctionTok{cutree}\NormalTok{(hc\_train, }\AttributeTok{k =} \DecValTok{5}\NormalTok{)}

\CommentTok{\# compute centroids for training clusters}
\NormalTok{centroids }\OtherTok{\textless{}{-}} \FunctionTok{aggregate}\NormalTok{(train, }\FunctionTok{list}\NormalTok{(cl\_train), mean)}
\NormalTok{centroids }\OtherTok{\textless{}{-}}\NormalTok{ centroids[,}\SpecialCharTok{{-}}\DecValTok{1}\NormalTok{]}

\CommentTok{\# distance from each test row to each centroid}
\NormalTok{dist\_to\_cent }\OtherTok{\textless{}{-}} \FunctionTok{matrix}\NormalTok{(}\ConstantTok{NA}\NormalTok{, }\AttributeTok{nrow =} \FunctionTok{nrow}\NormalTok{(test), }\AttributeTok{ncol =} \FunctionTok{nrow}\NormalTok{(centroids))}
\ControlFlowTok{for}\NormalTok{ (i }\ControlFlowTok{in} \DecValTok{1}\SpecialCharTok{:}\FunctionTok{nrow}\NormalTok{(centroids)) \{}
\NormalTok{  dist\_to\_cent[, i] }\OtherTok{\textless{}{-}} \FunctionTok{rowSums}\NormalTok{((test }\SpecialCharTok{{-}}\NormalTok{ centroids[i, ])}\SpecialCharTok{\^{}}\DecValTok{2}\NormalTok{)}
\NormalTok{\}}

\NormalTok{pred\_test }\OtherTok{\textless{}{-}} \FunctionTok{apply}\NormalTok{(dist\_to\_cent, }\DecValTok{1}\NormalTok{, which.min)}

\CommentTok{\# clustering test set directly}
\NormalTok{d\_test }\OtherTok{\textless{}{-}} \FunctionTok{dist}\NormalTok{(test)}
\NormalTok{hc\_test }\OtherTok{\textless{}{-}} \FunctionTok{agnes}\NormalTok{(d\_test, }\AttributeTok{method =} \StringTok{"ward"}\NormalTok{)}
\NormalTok{true\_test }\OtherTok{\textless{}{-}} \FunctionTok{cutree}\NormalTok{(hc\_test, }\AttributeTok{k =} \DecValTok{5}\NormalTok{)}

\CommentTok{\# stability}
\FunctionTok{adjustedRandIndex}\NormalTok{(pred\_test, true\_test)}
\end{Highlighting}
\end{Shaded}

\begin{verbatim}
## [1] 0
\end{verbatim}

After splitting the data and then testing the assignments, there is an
ARI of 0 which indicates that the clusters are formed no better than
random chance. I believe though that this is due to my method of
splitting and testing the data as the full data method earlier shows
clear differences between clusters.

Based on the results from the previous dendrogram, the red left cluster
that contains cereals such as 100\% natural bran, raisin nut bran, and
great grains would be best for a healthy rotation of cereals. The
cluster contains enough cereals for variety and contains cereals that
have higher than average fiber and protein with lower than average
sugar.

Also, to obtain this information normalization is required. Columns such
as calories contain a very high range compared to protein which is much
smaller. Without normalization, the columns similar to calories would
skew the data.

\end{document}
